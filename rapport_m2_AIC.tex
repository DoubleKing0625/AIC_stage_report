%% Template pour rapport de stage du master AIC (Apprentissage,
%% Information et Contenu), Universit� Paris-Saclay
%% Template d'origine :
%% Copyright (C) 2008 Johan Oudinet <oudinet@lri.fr>
%%
%% Permission is granted to make and distribute verbatim copies of
%% this manual provided the copyright notice and this permission notice
%% are preserved on all copies.
%%
%% Permission is granted to process this file through TeX and print the
%% results, provided the printed document carries copying permission
%% notice identical to this one except for the removal of this paragraph
%% (this paragraph not being relevant to the printed manual).
%%
%% Permission is granted to copy and distribute modified versions of this
%% manual under the conditions for verbatim copying, provided that the
%% entire resulting derived work is distributed under the terms of a
%% permission notice identical to this one.
%%
%% Permission is granted to copy and distribute translations of this manual
%% into another language, under the above conditions for modified versions,
%% except that this permission notice may be stated in a translation
%% approved by the Free Software Foundation
%%

\documentclass[oneside]{memoir}
\input{definition}


\usepackage[english]{inputenc}
\usepackage{epsfig}
\usepackage{acronym}
\usepackage{amssymb}
\usepackage{amsmath}
\usepackage{amsfonts}
\usepackage{pgf}
\usepackage{tikz}
\usepackage{setspace}%
\usepackage{hhline}
\usepackage[colorlinks]{hyperref}
\usepackage[all]{hypcap}
\usepackage{algorithm, algorithmic}
\usepackage[english]{babel}
\usetikzlibrary{shapes,arrows,automata,backgrounds}
\usepackage{wrapfig}
\usepackage{microtype}
\usepackage{multicol}
\usepackage{multirow}
\usepackage{cite}
\usepackage{indentfirst}

\usepackage{graphicx}
\usepackage{float}
\usepackage{caption}
\usepackage{subfigure}
\usepackage{listings}
% \usepackage[T1]{fontenc}

\lstset{
 columns=fixed,
 %numbers=left,                                        % 在左侧显示行号
 numberstyle=\tiny\color{gray},                       % 设定行号格式
 frame=none,                                          % 不显示背景边框
 backgroundcolor=\color[RGB]{245,245,244},            % 设定背景颜色
 keywordstyle=\color[RGB]{40,40,255},                 % 设定关键字颜色
 numberstyle=\footnotesize\color{darkgray},
 commentstyle=\color[RGB]{0,96,96},                % 设置代码注释的格式
 stringstyle=\rmfamily\slshape\color[RGB]{128,0,0},   % 设置字符串格式
 showstringspaces=false,                              % 不显示字符串中的空格
 language=c++,                                        % 设置语言
}


% \input{french_stuff}
%% Page layout
\setcounter{secnumdepth}{3}
\setcounter{tocdepth}{1}

\setlength{\oddsidemargin}{1cm}
\setlength{\evensidemargin}{1.5cm}
\setlength{\topmargin}{0cm}
\setlength{\textheight}{21cm}
\setlength{\textwidth}{15cm}
\setlength{\parindent}{2em}
% \setlength{\beforeschapskip}{-2cm}

\newcommand{\HRule}{\rule{\linewidth}{0.5mm}}

\begin{document}

\begin{titlingpage}

\begin{center}

%%%%%%%%%%%%%%%%%%%%%%%%%%%%%%%%%%%%%%%%%%%%%%%%%%%%%%%%%%%%%%%%%%%%%%%%%
% LOGOS:
%
\includegraphics[height=2cm]{logos/upsay.pdf} \hfill % Paris-Saclay: Don't remove !
\includegraphics[height=2cm]{logos/ups.png} % Your school inside Paris-Saclay
\\[1.5cm]


%%%%%%%%%%%%%%%%%%%%%%%%%%%%%%%%%%%%%%%%%%%%%%%%%%%%%%%%%%%%%%%%%%%%%%%%%
% Your specific logos (labs, companies, .... )
% \begin{minipage}{0.2\textwidth}
%   \centering
%   \begin{flushright}
%     \includegraphics[width=1.6\textwidth]{logos/face++.jpeg}
%   \end{flushright}
% \end{minipage}


\begin{figure}[htbp]
  \centering
  \includegraphics[width=3in]{logos/face++.jpeg}
\end{figure}


\vspace{2cm}
\textsc{\Large Internship of Research Master 2 in Computer Science}\\[0.5cm]


% Title
\HRule \\[0.4cm]
{ \huge \bfseries Work on COCO 2018 Keypoint Detection Task}\\[0.4cm]

\HRule \\[1.5cm]

% Author and supervisor
\begin{minipage}{0.4\textwidth}
\begin{flushleft} \large
\emph{Author :}\\
Qixiang \textsc{PENG}
\end{flushleft}
\end{minipage}
\begin{minipage}{0.4\textwidth}
\begin{flushright} \large
\emph{Stage chief :} \\
Dr. Gang \textsc{YU}
\end{flushright}
\end{minipage}
\vfill
\emph{Host organization : }
Megvii Research


\vfill



% Bottom of the page
{Secretariat - tel : 01 69 15 81 58\\
Email Address: alexandre.verrecchia@u-psud.fr\\
}
\end{center}

\end{titlingpage}


\pagenumbering{roman}

\tableofcontents

\newpage
\thispagestyle{empty}%
\newpage

\chapter*{Abstract}

This report summarizes my internship in Megvii Research: Work on COCO 2018 Keypoint Detection Task\footnote{ More details about this challenge can be found in https://competitions.codalab.org/competitions/12061}.
This challenge is designed to push the state of the art in multi-person pose estimation.

The topic of multi-person pose estimation has been largely improved recently, especially with the development of convolutional neural network.
However, there still exist a lot of challenging cases, such as occluded keypoints, invisible keypoints and complex background.

Nowadays, two solutions are adopted widely: Bottom-Up approaches and Top-Down approaches.
In this challenge, our team proposed a novel multi-stage top-down method.
More specifically, each stage is based on the Resnet and the later stage will pay more attention to the harder pose samples.

Based on the proposed algorithm, we achieve state-of-art results on the COCO keypoint benchmark, with average
precision at $77.8\%$ mAp on the COCO test-dev dataset and $76.4\%$ on the COCO test-challenge dataset, which is a $4.8\%$ mAp
relative improvement compared with $73.0\%$ from the COCO 2017 keypoint challenge.

\section*{Keywords}
COCO 2018 Keypoint Detection, human pose estimation, multi-stage, top-down.

%\pagebreak

\pagenumbering{arabic}
\setcounter{page}{1}

%% Copyright (C) 2008 Johan Oudinet <oudinet@lri.fr>
%%
%% Permission is granted to make and distribute verbatim copies of
%% this manual provided the copyright notice and this permission notice
%% are preserved on all copies.
%%
%% Permission is granted to process this file through TeX and print the
%% results, provided the printed document carries copying permission
%% notice identical to this one except for the removal of this paragraph
%% (this paragraph not being relevant to the printed manual).
%%
%% Permission is granted to copy and distribute modified versions of this
%% manual under the conditions for verbatim copying, provided that the
%% entire resulting derived work is distributed under the terms of a
%% permission notice identical to this one.
%%
%% Permission is granted to copy and distribute translations of this manual
%% into another language, under the above conditions for modified versions,
%% except that this permission notice may be stated in a translation
%% approved by the Free Software Foundation
%%
\chapter{Introduction to Company and Team}
\label{sec:intro}

\section{Introduction to Company}
\label{sec:isauriam}

Founded in October 2011, Megvii is an Articial Intelligence company specialized in providing enterprises and developers with intelligent solutions and data services,
and is dedicated to the mission of “Create machines that can see and think”.
With the "cloud + end" system of Megvii Cloud and Megvii SensorNet as its core products, Mevgii has successfully offered solutions for over 800 enterprises in nance, security,
office, real estate and other business sectors.

Megvii holds more than 350 domestic and international patents. Over seven years of development, Megvii has gathered a workforce of over 1000 people, among whom more than 70\% are R&D stas. The core team of Megvii is composed of top geeks
who are alumni of universities like Tsinghua, Columbia, Oxford, etc., and adventurers formerly working for Google, Alibaba, Huawei and IBM.
Over 80 people in Megvii have been awarded golden prizes of informatics at national and international levels.
Research teams from Megvii have been and are holding the first places in more than ten international AI benchmarks.

The name ’Megvii’ is from mega vision, which means our work is concentrated on offering computer vision technologies that enable your applications to read and
understand the world better.
In fact, FACE++, the best product of megvii, now, is the biggest platform of face detection over the world.

Here is the link to official website: https://www.faceplusplus.com/


\section{Introduction to Team}
\label{sec:isauriam}

During the internship, I worked in Detection Team in Megvii Research.
The team leader is Gang YU\footnote{His google scholar is: https://scholar.google.com/citations?user=BJdigYsAAAAJ&hl=en}.

In general, our team is in charge of 4 main issues:
\begin{enumerate}
  \item{\textbf{Detection: }} Face Detection, Pedestrian/human Detection, Vehicle/Plate Detection, General Object Detection, Object Detection in Video, 3D Object detection (combined with Point Cloud)
  \item{\textbf{Segmentation: }} Semantic Segmentation, Instance Segmentation, Panoptic Segmentation, Video Segmentation, 3D Segmentation
  \item{\textbf{Skeleton: }} Human Pose Estimation, Hand Pose Estimation
  \item{\textbf{Action: }} Action Recognition in Video
\end{enumerate}

Our team has a solid technical accumulation, especially in the detection aspect. We have the winner solution of COCO2017 Detection:  MegDet\cite{peng2018megdet}.
From a product perspective, we built a small repo of imagenet base model for training and exploring models with less than 100M FLOPs.
In addition to Detection, our skeleton solution also took the first in the COCO2017 Human Pose competition: CPN\cite{chen2017cascaded}.
In terms of Segmentation, we also have some better work published. In addition, we have sufficient GPU resources, as well as very large internal data sets for exploring the upper-bound of various research tasks.




%%% Local Variables:
%%% mode: latex
%%% TeX-master: "rapportM2R"
%%% End:


%% Copyright (C) 2008 Johan Oudinet <oudinet@lri.fr>
%%
%% Permission is granted to make and distribute verbatim copies of
%% this manual provided the copyright notice and this permission notice
%% are preserved on all copies.
%%
%% Permission is granted to process this file through TeX and print the
%% results, provided the printed document carries copying permission
%% notice identical to this one except for the removal of this paragraph
%% (this paragraph not being relevant to the printed manual).
%%
%% Permission is granted to copy and distribute modified versions of this
%% manual under the conditions for verbatim copying, provided that the
%% entire resulting derived work is distributed under the terms of a
%% permission notice identical to this one.
%%
%% Permission is granted to copy and distribute translations of this manual
%% into another language, under the above conditions for modified versions,
%% except that this permission notice may be stated in a translation
%% approved by the Free Software Foundation
%%
\chapter{Presentation of the context of the task}
\label{sec:intro}
In this chapter, several brief presentations will be given to explain the context of the human pose estimation and describe the COCO dataset.

\section{Human Pose Estimation}
\label{sec:isauriam}
Localizing body parts for human body is a fundamental yet challenging task in computer vision, and it serves as an important basis for high-level vision tasks, e.g., activity
recognition\cite{yang2010recognizing, wang2013approach}, human re-identification\cite{zheng2017pose}, and human-computer interaction.
In general,a human pose estimation model aims to predict the 2D coordinates of different human parts given a 2D human image.
Achieving accurate localization, however, is difficult due to the highly articulated human body limbs, occlusion, change of viewpoint, and foreshortening.

Classical approaches tackling the problem of human pose estimation mainly adopt the techniques of pictorial structures \cite{fischler1973representation} or graphical models\cite{chen2014articulated}.
More specifically, the classical works\cite{andriluka2009pictorial, gkioxari2013articulated, sapp2013modec, johnson2011learning} formulate the problem of human keypoints estimation as a tree-structured or graphical model problem and predict keypoint locations based on hand-crafted features.
Recent works\cite{newell2016stacked, gkioxari2016chained, wei2016convolutional, insafutdinov2016deepercut} mostly rely on the development of convolutional neural network(CNN)\cite{lecun1998gradient, he2016deep}, which largely improve the performance of pose estimation.
And the rest of report also focuses on the solution based on CNN.

Nowadays there exists two main topics in human pose estimation: single person pose estimation and multi-person pose estimation. Obviously, multi-person is more challenging than single person pose estimaion but single person is the fundamentation for multi-person pose estimation, as shown in Figure.1.

\captionsetup[figure]{labelformat=empty}
\begin{figure}
  \centering
  \subfigure[Single person pose estimation]{
    \label{fig:subfig:a} %% label for first subfigure
    \includegraphics[width=5cm,height=4cm]{source/single.png}}
  \hspace{1in}
  \subfigure[Multi-person pose estimation]{
    \label{fig:subfig:b} %% label for second subfigure
    \includegraphics[width=5cm,height=4cm]{source/multi.png}}
  \caption{Figure 1:(a) is the example of single person pose estimation, only one person in a image. (b) is the example of multi-persion pose
  estimation. An image includes several peoples. We need to detect all the keypoints and group them into the right person ID.}
  \label{fig:1} %% label for entire figure
\end{figure}

\subsection{Single person Pose Estimation}
The works based on CNN usually adopts two methods: regress the coordinates of keypoints directly and regress the cofidence score map of keypoints. Toshev \textit{et al.} firstly introduce
CNN to solve pose estimation problem in the work of DeepPose\cite{toshev2014deeppose}, which proposes a cascade of CNN keypoint coordinate regressors to deal with pose estimation. Tompson \textit{et al.}\cite{tompson2014joint}
attempt to solve the problem by predicting heatmaps of keypoints using CNN and graphical models.
Using heatmap as the supervised label can provide more robust information, hence recently most of work focus on predicting heatmaps.
For example, latest works such as Wei \textit{et al.}\cite{wei2016convolutional} and Newell \textit{et al.}\cite{newell2016stacked} show great performance via generating the score map of keypoints using very
deep convolutional neural networks.

\subsection{Multi-person Pose estimation}
Multi-person pose estimation is gaining increasing popularity recently because of the high demand for the real-life applications.
However, multi-person pose estimation is challenging owing to occlusion, various gestures of individual persons and unpredictable interactions between different persons.
The approach of multi-person pose estimation is mainly divided into two categories: bottom-up approaches and top-down approaches.

\subsubsection{Bottom-Up Approaches}

Bottom-up approaches\cite{newell2017associative, insafutdinov2016deepercut, pishchulin2016deepcut} directly predict all keypoints at first and assemble them into full poses of all persons.
DeepCut\cite{pishchulin2016deepcut} interprets the problem of distinguishing different persons in an image as an Integer Linear Program (ILP) problem and partition
part detection candidates into person clusters.
Then the final pose estimation results are obtained when person clusters are combined with labeled body parts.
DeeperCut\cite{insafutdinov2016deepercut} improves DeepCut\cite{pishchulin2016deepcut} using deeper ResNet\cite{he2016deep} and employs image-conditioned pairwise terms to get better performance.
Zhe Cao \textit{et al.}\cite{cao2016realtime} map the relationship between keypoints into part affinity fields (PAFs) and assemble detected keypoints into different poses of people.
Newell \textit{et al.}\cite{newell2017associative} simultaneously produce score maps and pixel-wise embedding to group the candidate keypoints to different people to get final multi-person pose estimation.

\subsubsection{Top-down Approaches}
Top-down approaches\cite{huang2017coarse, papandreou2017towards, he2017mask} interpret the process of detecting keypoints as a twostage pipeline, that is, firstly locate and crop all persons from image,
and then solve the single person pose estimation problem in the cropped person patches.
Papandreou \textit{et al.}\cite{papandreou2017towards} predict both heatmaps and offsets of the points on the heatmaps to the ground truth location, and then uses the heatmaps with offsets to obtain the final predicted location of keypoints.
Mask-RCNN\cite{he2017mask} predicts human bounding boxes first and then crops the feature map of the corresponding human bounding box to predict human keypoints.
If top-down approach is utilized for multi-person pose estimation, a human detector as well as single person pose estimator is important in order to obtain a good performance.

\subsubsection{Human detection}
Human detection approaches are mainly guided by the RCNN family\cite{girshick2014rich, girshick2015fast, ren2015faster}, the upto-date detectors of which are\cite{lin2017feature, he2017mask}.
These detection approaches are composed of two-stage in general.
First generate boxes proposals based on default anchors, and then crop from the feature map and further refine the proposals to get the final boxes via R-CNN network.


\section{COCO dataset and metric}

COCO\cite{lin2014microsoft} is a large-scale object detection, segmentation, and captioning dataset\footnote{More details about COCO dataset can be found in http://cocodataset.org/}.
For different tasks like object detection, keypoint detection, stuff segmentaion, etc, COCO dataset can be seperated as different sub-datasets.
Here we only introduce the human keypoint dataset.

\subsection{Annotation format}

Each person(instance) annotations contains a series of fields. \textbf{1.} image\_id: Indicates which images this person belongs to. \textbf{2.} bbox: Indicates the location of this person in the image.
\textbf{3.} keypoints: a is a length 3*17 array, indicates the details of the keypoints. Each keypoint has a 0-indexed location x,y and a visibility flag v defined as v=0: not labeled (in which case x=y=0), v=1: labeled but not visible, and v=2: labeled and visible.

There are 17 keypoints: 0: nose, 1: left eye, 2: right eye, 3: left ear, 4: right ear, 5: left shoulder, 6: right shoulder, 7: left elbow, 8: right elbow, 9: left wrist, 10: right wrist, 11: left hip, 12: right hip, 13: left knee, 14: right knee, 15: left ankle, 16: right ankle.

\begin{lstlisting}
int a  = b;
\end{lstlisting}


%%% Local Variables:
%%% mode: latex
%%% TeX-master: "rapportM2R"
%%% End:


%% Copyright (C) 2008 Johan Oudinet <oudinet@lri.fr>
%%
%% Permission is granted to make and distribute verbatim copies of
%% this manual provided the copyright notice and this permission notice
%% are preserved on all copies.
%%
%% Permission is granted to process this file through TeX and print the
%% results, provided the printed document carries copying permission
%% notice identical to this one except for the removal of this paragraph
%% (this paragraph not being relevant to the printed manual).
%%
%% Permission is granted to copy and distribute modified versions of this
%% manual under the conditions for verbatim copying, provided that the
%% entire resulting derived work is distributed under the terms of a
%% permission notice identical to this one.
%%
%% Permission is granted to copy and distribute translations of this manual
%% into another language, under the above conditions for modified versions,
%% except that this permission notice may be stated in a translation
%% approved by the Free Software Foundation
%%
\chapter{Solution to COCO 2018 keypoint task}
\label{sec:solution}
Similar to\cite{he2017mask, papandreou2017towards}, our algorithm adopts the top-downpipeline:
a human detector is first applied on the image to generate a set of human bounding-boxes and detailed localization
of the keypoints for each person can be predicted by a single-person skeleton estimator.
In addition, I use GAN to generate new train data. The extended train-set makes model more robust.

\section{Human detector}
We adopt the state-of-art object detector algorithms based on FPN\cite{lin2017feature}. ROIAlign from Mask RCNN\cite{he2017mask} is
adopted to replace the ROIPooling in FPN.
To train the object detector, all eighty categories from the COCO dataset are utilized during the training process but only the boxes of
human category is used for our multi-person skeleton task.

In our pipeline, we need a object detector which can find out the candidate human bounding-boxes as many as possible,
not the one which can predict the human bounding-boxes very accurately.
In another words, we need a object detector with high recall, not high precision.

\section{Human Pose Estimator}
Before starting the discussion of our algorithm, I first briefly review the design structure for the single person pose
estimator based on each human bounding box.
\subsection{Stacked Hourglass}
Stacked hourglass\cite{newell2016stacked}, which is a prevalent method for pose estimation,
stacks eight hourglasses which are down-sampled and up-sampled modules with residual connections to enhance the pose estimation performance.

The design of the hourglass is motivated by the need to capture information at every scale.
While local evidence is essential for identifying features like faces and hands, a final pose estimate requires a coherent understanding of the full body.
The person’s orientation, the arrangement of their limbs, and the relationships of adjacent joints are among the many cues that are best recognized at different scales in the image.
The hourglass is a simple, minimal design that has the capacity to capture all of these features and bring them together to output pixel-wise predictions.

Then the network is handled further by stacking multiple hourglasses end-to-end, feeding the output of one as input into the next.
This provides the network with a mechanism for repeated bottom-up, top-down inference allowing for reevaluation of initial estimates and features across the whole image.
The key to this approach is the prediction of intermediate heatmaps upon which a loss can be applyed.
Predictions are generated after passing through each hourglass where the network has had an opportunity to process features at both local and global contexts.
Subsequent hourglass modules allow these high level features to be processed again to further evaluate and reassess higher order spatial relationships.


\captionsetup[figure]{labelformat=empty}
\begin{figure}
  \centering
  \subfigure[One Hourglass]{
    \label{fig:subfig:a} %% label for first subfigure
    \includegraphics[width=6cm,height=4cm]{source/hourglass_single.png}}
  \hspace{1in}
  \subfigure[Stacked Hourglasses]{
    \label{fig:subfig:b} %% label for second subfigure
    \includegraphics[width=6cm,height=4cm]{source/hourglass_total.png}}
  \caption{Figure 3: Hourglass Module. (a): The illustration of a single “hourglass” module. Each box in the figure corresponds
to a residual module. The number of features is consistent across the whole hourglass. (b) The illustration of multiple stacked hourglass modules
which allow for repeated bottom-up, top-down inference}
  \label{fig:1} %% label for entire figure
\end{figure}

\subsection{Cascaded Pyramid Network}
CPN\cite{chen2017cascaded} thought that stacking two hourglasses is sufficient to have a comparable performance
compared with the eight-stage stacked hourglass module. Hence, CPN involves two sub-networks: GlobalNet and RefineNet.

GlobalNet is based on the ResNet backbone. The last residual blocks of different conv features conv2∼5 are denoted as $C_{2}$, $C_{3}$, ..., $C_{5}$ respectively.
3 × 3 convolution filters are applied on $C_{2}$, ..., $C_{5}$ to generate the heatmaps for keypoints.
The shallow features like $C_{2}$ and $C_{3}$ have the high spatial resolution for localization but low semantic information for recognition.
On the other hand, deep feature layers like $C_{4}$ and $C_{5}$ have more semantic information but low spatial resolution due to strided convolution (and pooling).
Thus, usually an U-shape structure is integrated to maintain both the spatial resolution and semantic information for the feature layers.
GlobalNet can effectively locate the keypoints like eyes but may fail to precisely locate the position of hips.
The localization of keypoints like hip usually requires more context information and processing rather than the nearby appearance feature.

Based on the feature pyramid representation generated by GlobalNet, a RefineNet to explicitly address the
“hard” keypoints was attached.
In order to improve the efficiency and keep integrity of information transmission,
the RefineNet transmits the information across different levels and finally integrates the informations of different levels via upsampling
and concatenating as HyperNet\cite{kong2016hypernet}. In addition, more bottleneck blocks are stacked into deeper layers,
whose smaller spatial size achieves a good trade-off between effectiveness and efficiency.

\captionsetup[figure]{labelformat=empty}
\begin{figure}[htbp]
  \centering
  \includegraphics[width=16cm,height=4cm]{source/cpn.png}
  \caption{Figure 4: Cascaded Pyramid Network. Output heatmaps from different features. The green dots means the groundtruth location of keypoints.
  GlobalNet handles the easy case like left eye well.
  RefineNet integrates more information to locate the hard case, like occluded left hip.}
\end{figure}

\subsection{Our Pose Estimator}
Motivated by the works\cite{newell2016stacked, chen2017cascaded} described above, we propose an effective and efficient network to address the problem of pose estimation.
As shown in Figure 5, our model is a multi-stage network.

\captionsetup[figure]{labelformat=empty}
\begin{figure}[htbp]
  \centering
  \includegraphics[width=16cm,height=6cm]{source/multi-stage.png}
  \caption{Figure 5: Our multi-stage network. Each stage is a U-shape structure based on the variant resnet.
  Different stages focus on diffferent poses. The later stage studies the harder pose.}
\end{figure}

\subsubsection{Singe Stage}

\textbf{Every single stage is a U-shape structure}.

As we all known, classification task needs semantic information and spatial information is necessary for location task.
The keypoint task contains both classification and location task because we need to not only classifer different body parts but also give precise coordinates.
In other words, we need both semation information and spatial information.
This is why we adopt U-shape structure.

The U-shape structure is based on the ResNet backbone. The last residual blocks of different conv features conv2∼5 are denoted as $C_{2}$, $C_{3}$, ..., $C_{5}$ respectively, like the cuboids with full line in Figure 5.

In the down-sampling procedure, the resolution of feature map zooms out 5 times(conv1~5).
For example, if the size of input was $256*192$, the lowest resolution could be $8*6$.
Hence, every pixel in $8*6$ size feature map owns a huge receptive field and the $8*6$ size feature map has high-dimensinal semantic information but lacks spatial information.
On the contrary, the high solution feature map has more spatial information and less semantic information.

And then the up-sampling procedure will be passed.
With a coarser-resolution feature map, we upsample the spatial resolution by a factor of 2 (using bilinear interpolation for accuracy).
The upsampled map is then merged with the corresponding down-sampling map (which undergoes a $1*1$ convolutional layer to reduce channel dimensions) by element-wise addition.
This process is iterated until the finest resolution map is generated.
To start the iteration, we simply attach a $1×1$ convolutional layer on $C_{5}$ to produce the coarsest resolution map.
Afterwards, we append a $1*1$ convolution on each merged map to generate the final feature map, which is to reduce the aliasing effect of upsampling and to reduce the dimensions(numbers of channels) to 256.
This final set of feature maps is called $\{P_{2}, P_{3}, P_{4}, P_{5}\}$, corresponding to $\{C_{2}, C_{3}, C_{4}, C_{5}\}$
that are respectively of the same spatial sizes.

Finally, every final feature maps will pass 2 Conclusion layes($3*3$ and $1*1$) to generate the 17 dimensinal heapmaps.

\captionsetup[figure]{labelformat=empty}
\begin{figure}[htbp]
  \centering
  \includegraphics[width=16cm,height=6cm]{source/single-stage.png}
  \caption{Figure 6: Details of U-shape structure for single stage}
\end{figure}

\textbf{Intermediate Supervision}

The U-shape architecture provides a top-down, bottom-up inference allowing for generation of heatmaps with different resolution.
Hence, we can apply a loss upon the prediction of intermediate heatmaps.
Predictions are generated after passing through each final feature map $\{P_{2}, P_{3}, P_{4}, P_{5}\}$ where the network has had an opportunity to process features at both local and global contexts.
This is similar to other pose estimations methods that have demonstrated strong performance with multiple iterative intermediate supervision\cite{wei2016convolutional, carreira2016human}.

\textbf{Every single stage is a variant Resnet}.

Nowadays, Resnet\cite{he2016deep} is the most common basemodel in all kinds of Computer-vision task.
The core idea of Resnet is the residual learning which can avoid the gradient vanishing effectively.
Hence, Resnet can be designed very deeply, like resnet50, resnet101, resnet152.
\[ y = F(x, \{W_{i}\}) + x ...................(2)\]
Eqn 1 explained the residual learning. Here $x$ and $y$ are the input and output vectors of the layers considered.
The function $F(x, \{W_{i}\})$ represents the residual mapping to be learned.

\captionsetup[figure]{labelformat=empty}
\begin{figure}
  \centering
  \subfigure[One Hourglass]{
    \label{fig:subfig:a} %% label for first subfigure
    \includegraphics[width=6cm,height=4cm]{source/residual_learning.png}}
  \hspace{1in}
  \subfigure[Stacked Hourglasses]{
    \label{fig:subfig:b} %% label for second subfigure
    \includegraphics[width=6cm,height=4cm]{source/residual_block.png}}
  \caption{Figure 7: Residual block. (a): The residual-block for residual learning. (b) The realization of residual block in Resnet50, 101, 152}
  \label{fig:1} %% label for entire figure
\end{figure}

As mentioned before, keypoint detection task needs both semantic information and spatial information.
But are they same important?
We did several experiments using res50, res101 and res152 as single stage network respectively.
Single stage has mAp $73.3\%$ with res50, $73.7\%$ with res101, $73.8\%$ with res152.
Compared with res50, res101 got $0.4\%$ mAp gain.
However, res152 only got $0.1\%$ more mAp than res101.
As shown in Figure 8, the difference better res50, res101 and res152 is the number of residual blocks in Conv4.
Conv4 learns more semantic information, and increasing the complexity in Conv4 doesn't seem to work.
Hence, we got a conclusion: \textbf{for keypoint detection task, spatial information is more important than semantic information.}

Then we designed a variant Resnet.
In normal Res101, The number of residual block of different Conv stage is $\{3, 4, 23, 3\}$.
Our variant Res101 adopts $\{4, 8, 16, 3\}$, which maintains same FLOPs as normal Res101 but adds more complexity in Conv2 and Conv3.
And it has $74.1\%$ mAp, more $0.4\%$ mAp than normal Res101 in the case of the same complexity.


\captionsetup[figure]{labelformat=empty}
\begin{figure}[htbp]
  \centering
  \includegraphics[width=16cm,height=6cm]{source/resnet.png}
  \caption{Figure 8: Different Resnet Architectures}
\end{figure}

\subsubsection{Multi-Stage}

\textbf{Hard pose samples are really rare.}

In keypont detection, the object keypoint similarity (OKS), seen in Eqn 1, is required to define positives and negatives.
At beginning, we trained a one-stage model and did the failure case analyse.
We found that most normal poses can be handled well, but a few hard poses like Parkour or playing skateboard are detected with low OKS.
Two main factors are responsible for this: 1)hard poses are inherently challenging  and 2) overfitting for normal poses during training, due to lack of hard poses.
Hence, motivated by cascade r-cnn\cite{cai2017cascade}, a multi-stage keypoint estimator architecture is proposed to the second factor.
It consists of a sequence of single estimator as mentionde before trained with increasing OKS thresholds, to be sequentially more selective against close false positives.
The estimators are trained stage by stage, leveraging the observation that the output of a estimator is a good distribution for training the next higher quality estimator.

For example, if we take 3 stage architecture, the input of first stage are all samples,
the input of second stage are those whose OKS is smaller than 0.8 accroding to the output of first stage,
and the input of third stage are those whose OKS is smaller than 0.6 accroding to the output of second stage.
The resampling of progressively guarantees that the later stage pays morr attention to the harder pose, reducing
the overfitting problem.

\textbf{We don't need pre-train a model using Imagenet.}

Since R-cnn\cite{girshick2014rich}, we always initialize our basemode by the parameters pre-trained on Imagenet, then finu-tune the model by the specific dataset.
It seems like a common sense, which no one doubts.
Howerver, is that really suitable for keypoint detection?
\textbf{Our answer is no.}

We did several comparative experiments, Table 2 shows the results.

\captionsetup[table]{labelformat=empty}
\begin{table}[!hbp]
  \centering
  \begin{tabular}{|c|c|c|c|c|}
  \hline
            & 3xres50 & 4xres50 & 2xres101 & 2xresInc101  \\
  \hline
  Initialization using Gaussian & 77.5 & 77.8 & 77.0 & 77.5 \\
  \hline
  Initialization using pre-trained parameters & 76.9 & 77.0 & 76.2 & 76.9 \\
  \hline
  \end{tabular}
  \caption{Table 2: comparative experiments results for initialization problem.}
\end{table}

We can find that initialization using Gaussian is generally better than initialization using pre-trained parameters.
Hence, we get a conclusion: \textbf{Imagenet task is a pure classification task, which prefers to semantic information.
This is why using parameters pre-trainded on Imagenet for initialization results in a worse performance.
The best solution is that we pre-train our model in a large auxiliary keypint dataset.}





%%% Local Variables:
%%% mode: latex
%%% TeX-master: "rapportM2R"
%%% End:


%% Copyright (C) 2008 Johan Oudinet <oudinet@lri.fr>
%%
%% Permission is granted to make and distribute verbatim copies of
%% this manual provided the copyright notice and this permission notice
%% are preserved on all copies.
%%
%% Permission is granted to process this file through TeX and print the
%% results, provided the printed document carries copying permission
%% notice identical to this one except for the removal of this paragraph
%% (this paragraph not being relevant to the printed manual).
%%
%% Permission is granted to copy and distribute modified versions of this
%% manual under the conditions for verbatim copying, provided that the
%% entire resulting derived work is distributed under the terms of a
%% permission notice identical to this one.
%%
%% Permission is granted to copy and distribute translations of this manual
%% into another language, under the above conditions for modified versions,
%% except that this permission notice may be stated in a translation
%% approved by the Free Software Foundation
%%
\chapter{Experiments}
Our overall pipeline follows the top-down approach for estimating multiple human poses.
Firstly, we apply a stateof-art bounding detector to generate human proposals.
For each proposal, we assume that there is only one main person in the cropped region of proposal and then applied the pose estimating network to generate the final prediction.
In this chapter, I will discuss more details of our methods based on experiment results.

\section{Experimental Setup}

\subsection{Cropping Strategy}
For each human detection box, the box is extended to a fixed aspect ratio, e.g., height : width = 256 : 192, and then we crop from images without distorting
the images aspect ratio.
Finally, we resize the cropped image to a fixed size of height 384 pixels and width 288 pixels by default.

\subsection{Data Augmentation Strategy}
Data augmentation is critical for the learning of scale invariance and rotation invariance.
After cropping from images, if one case has more than 8 visible keypoint, we apply demi part augmentation with a probability of 0.3.
Then we apply random flip, random rotation $(−45^{\circ} \sim +45^{\circ})$ and random scale $(0.7 \sim 1.35)$.

\subsection{Training Details}
All models of pose estimation are trained using adam algorithm with an initial learning rate of 5e-4.
We train the model for 400 epochs and each epoch contains 90000 samples.
Note that we decrease the learning rate linearly decrease during the whole training process.
We use a weight decay of 1e5 and the training batch size is 32.
Batch normalization is used in our network.
Generally, the training of 3 ResNet-50-based models takes about 5 days on eight NVIDIA Titan X Pascal GPUs.

\subsection{Testing Details}
In order to minimize the variance of prediction, we apply a gaussian filter on the predicted heatmaps.
Following the same techniques used in \cite{newell2016stacked}, we also predict the pose of the corresponding flipped image and average the heatmaps to get the final prediction; a quarter
offset in the direction from the highest response to the second
highest response is used to obtain the final location of
the keypoints.
Rescoring strategy is also used in our experiments.
Different from the rescoring strategy used in \cite{papandreou2017towards},
the product of boxes’ score and the average score of all keypoints
is considered as the final pose score of a person instance.

\section{Ablation Experiment}
In this section, I will validate the effectiveness of our
network from various aspects. Unless otherwise specified,
all experiments are evaluated on MS COCO minival dataset
in this section. The input size of all models is 256×192
and the same data augmentation is adopted.

\subsection{Multi-stage}
A single stage with Res50 is adopted as our baseline. From Table 3, there exists a trade-off between the number of stage and the complexity of one stage because of the computation resource.
Finally, we find 4xres50 is the best choice.  Note that 4xres50 is used in all the next experiments.

\captionsetup[table]{labelformat=empty}
\begin{table}[!hbp]
  \centering
  \begin{tabular}{|c|c|c|c|c|c|c|c|}
  \hline
            & baseline & 2xres50 & 3xres50 & 4xres50 & 2xres101 & 2xresInc101 & 5xres38 \\
  \hline
  AP(OKS) & 73.3 & 76.6 & 77.3 & 77.8 & 77.0 & 77.5 & 77.5 \\
  \hline
  \end{tabular}
  \caption{Table 3: experiments results for multi-stage.}
\end{table}

\subsection{Online Hard Keypoints Mining}
Here I discuss the losses used in our network. In detail,
the loss function of single stage is L2 loss of all annotated
keypoints while the Conv2 tries learning the hard keypoints,
that is, we only punish the top M(M < N) keypoint
losses out of N (the number of annotated keypoints in one
person, say 17 in COCO dataset). The effect of M is shown
in Table 4. For M = 8, the performance of second stage
achieves the best result for the balanced training between
hard keypoints and simple keypoints.

\captionsetup[table]{labelformat=empty}
\begin{table}[!hbp]
  \centering
  \begin{tabular}{|c|c|c|c|c|c|c|}
  \hline
  M          & 6 & 8 & 10 & 12 & 14 & 17 \\
  \hline
  AP(OKS) & 77.4 & 77.8 & 77.5 & 77.5 & 77.1 & 77.0 \\
  \hline
  \end{tabular}
  \caption{Table 4: Comparison of different hard keypoints number in online hard keypoints mining.}
\end{table}

\subsection{Data Pre-processing}
The size of cropped image is a important factor to the performance
of keypoints detection. As Table 5 illustrates, it’s
worth noting that the input size 256×192 actually works as
well as 256×256 which costs more computations of almost
2G FLOPs using the same cropping strategy.
As the input size of the cropped images increases, more location details
of human keypoints are fed into the network resulting in a
large performance improvement.
Additionally, online hard
keypoints mining works better when the input size of the
crop images is enlarged by improving 1 point on 384 × 288
input size.

\captionsetup[table]{labelformat=empty}
\begin{table}[!hbp]
  \centering
  \begin{tabular}{|c|c|c|}
  \hline
  Models          & Input Size & AP(OKS)   \\
  \hline
  4res50* & 256x192 & 76.2 \\
  \hline
  4res50 & 256x192 & 76.9  \\
  \hline
  4res50* & 384x288 & 77.4  \\
  \hline
  4res50 & 384x288 & 77.8  \\
  \hline
  \end{tabular}
  \caption{Table 5: Comparison of models of different input size. 4xres50* indicates 4xres50 without online hard keypoints mining.}
\end{table}

At the end, we use ensemble on the heatmap level to get the final and best results.

\captionsetup[table]{labelformat=empty}
\begin{table}[!hbp]
  \centering
  \begin{tabular}{|c|c|c|c|}
  \hline
  Models          & AP-minival & AP-dev & AP-challenge  \\
  \hline
  4res50 & 77.8 & 74  & - \\
  \hline
  4res50+ & 80 & 77.8 & 76.2  \\
  \hline
  \end{tabular}
  \caption{Table 6: Comparison of results on the minvival dataset and the corresponding results on test-dev or test-challenge of the COCO
dataset. “+” indicates ensembled model.}
\end{table}


%%% Local Variables:
%%% mode: latex
%%% TeX-master: "rapportM2R"
%%% End:


%% Copyright (C) 2008 Johan Oudinet <oudinet@lri.fr>
%%
%% Permission is granted to make and distribute verbatim copies of
%% this manual provided the copyright notice and this permission notice
%% are preserved on all copies.
%%
%% Permission is granted to process this file through TeX and print the
%% results, provided the printed document carries copying permission
%% notice identical to this one except for the removal of this paragraph
%% (this paragraph not being relevant to the printed manual).
%%
%% Permission is granted to copy and distribute modified versions of this
%% manual under the conditions for verbatim copying, provided that the
%% entire resulting derived work is distributed under the terms of a
%% permission notice identical to this one.
%%
%% Permission is granted to copy and distribute translations of this manual
%% into another language, under the above conditions for modified versions,
%% except that this permission notice may be stated in a translation
%% approved by the Free Software Foundation
%%
\chapter{Conclusion}
I was mainly responsible for the COCO 2018 keypoint competition during this internship.
The time-line is like below:
\begin{enumerate}
  \item April $\sim$ May: Get similar to Human Pose Estimation problem and review related works and codes.
  \item May $\sim$ June: Think that dataset is a choke point and more data is necessary. Try to generate train data using GAN\cite{goodfellow2014generative}.
  \item June $\sim$ Mid-July: Design a novel network.
  \item Mid-July $\sim$ Mid-August(Deadline of Competition): Try to ensemble and fine-tune hyper-parameters.
\end{enumerate}

Finally, we proposed a novel multi-stage network which can learn both normal poses and hard poses well.
Based on the proposed algorithm, we achieved state-of-art results on the COCO keypoint benchmark, with average
precision at $77.8\%$ mAp on the COCO test-dev dataset and $76.4\%$ on the COCO test-challenge dataset, which is a $4.8\%$ mAp
relative improvement compared with $73.0\%$ from the COCO 2017 keypoint challenge.

\textbf{Our team was invited to make a presentation in ECCV2018, September 8 - 14 in Munich, Germany}\footnote{More details about the conference can be found in https://eccv2018.org/}.

Here, I want to record some feeling about this competition experience.
There are two main points:
\begin{enumerate}
  \item \textbf{The hardest part is the beginning and ending period.} At the beginning, I was new to Human Pose Estimation problem totally.
  At that time, leader requested me to do a quick review of the recent related works and make a presentation.
  I read papers every day and studied the source codes, but improve myself rapidly.
  In the last few weeks, all the ideas have been tried and our results on COCO benchmark have been stalled.
  I remember clearly that I was particularly anxious every day during that time.
  \item \textbf{Preparing the competition is totally different from doing research work.}
  In my opinion, when doing research, we should believe that one idea is useful and stick to it.
  But when preparing the competition, we may try many ideas and each idea has a end time point in order to avoid risk.
  For example, throughout the whole May, I have tried using GAN to generate new train samples. However it didn't work finally.
  So we started another attempt even though we all thought generating train data is a meaningful work. 
\end{enumerate}


%%% Local Variables:
%%% mode: latex
%%% TeX-master: "rapportM2R"
%%% End:


%% Copyright (C) 2008 Johan Oudinet <oudinet@lri.fr>
%%
%% Permission is granted to make and distribute verbatim copies of
%% this manual provided the copyright notice and this permission notice
%% are preserved on all copies.
%%
%% Permission is granted to process this file through TeX and print the
%% results, provided the printed document carries copying permission
%% notice identical to this one except for the removal of this paragraph
%% (this paragraph not being relevant to the printed manual).
%%
%% Permission is granted to copy and distribute modified versions of this
%% manual under the conditions for verbatim copying, provided that the
%% entire resulting derived work is distributed under the terms of a
%% permission notice identical to this one.
%%
%% Permission is granted to copy and distribute translations of this manual
%% into another language, under the above conditions for modified versions,
%% except that this permission notice may be stated in a translation
%% approved by the Free Software Foundation
%%
\chapter{Acknowledgement}

At the end of everthing, I would like to point out that i couldn't finish this internship
successfully without someones, and here I give my most sincere thanks to them.

Firstly, I will express thanks to Gang YU, my team leader. It's him
that taught me the detailed knowledge of computer vision and deep learing, like how to design a network , how to write the code using framework, how to train a model, etc.
He also explained papers to me clearly and carefully.
In fact, he leads me into the deep of CV\&DL domain.

Next, I would like to thank for Zhicheng WANG, my virtual team leader.
During all the COCO competition, he gave me lots of help, like correcting my wrong opinions and reviewing my codes.
We argued about the experiment results and make the plan together.
These experience have brought me tremendous progress.

Finally but with same importance, my colleagues helped me a lot when I encountered
some problems about software or hardware, I will always appreciate that.

%%% Local Variables:
%%% mode: latex
%%% TeX-master: "rapportM2R"
%%% End:


\bibliographystyle{plain}
\bibliography{rapport_m2_AIC}

\appendix

% \input{appendixA}

\end{document}


%%% Local Variables:
%%% mode: latex
%%% TeX-master: "rapportM2R"
%%% End:
