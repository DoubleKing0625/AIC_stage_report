%% Copyright (C) 2008 Johan Oudinet <oudinet@lri.fr>
%%
%% Permission is granted to make and distribute verbatim copies of
%% this manual provided the copyright notice and this permission notice
%% are preserved on all copies.
%%
%% Permission is granted to process this file through TeX and print the
%% results, provided the printed document carries copying permission
%% notice identical to this one except for the removal of this paragraph
%% (this paragraph not being relevant to the printed manual).
%%
%% Permission is granted to copy and distribute modified versions of this
%% manual under the conditions for verbatim copying, provided that the
%% entire resulting derived work is distributed under the terms of a
%% permission notice identical to this one.
%%
%% Permission is granted to copy and distribute translations of this manual
%% into another language, under the above conditions for modified versions,
%% except that this permission notice may be stated in a translation
%% approved by the Free Software Foundation
%%
\chapter{Conclusion}
I was mainly responsible for the COCO 2018 keypoint competition during this internship.
The time-line is like below:
\begin{enumerate}
  \item April $\sim$ May: Get similar to Human Pose Estimation problem and review related works and codes.
  \item May $\sim$ June: Think that dataset is a choke point and more data is necessary. Try to generate train data using GAN\cite{goodfellow2014generative}.
  \item June $\sim$ Mid-July: Design a novel network.
  \item Mid-July $\sim$ Mid-August(Deadline of Competition): Try to ensemble and fine-tune hyper-parameters.
\end{enumerate}

Finally, we proposed a novel multi-stage network which can learn both normal poses and hard poses well.
Based on the proposed algorithm, we achieved state-of-art results on the COCO keypoint benchmark, with average
precision at $77.8\%$ mAp on the COCO test-dev dataset and $76.4\%$ on the COCO test-challenge dataset, which is a $4.8\%$ mAp
relative improvement compared with $73.0\%$ from the COCO 2017 keypoint challenge.

\textbf{Our team was invited to make a presentation in ECCV2018, September 8 - 14 in Munich, Germany}\footnote{More details about the conference can be found in https://eccv2018.org/}.

Here, I want to record some feeling about this competition experience.
There are two main points:
\begin{enumerate}
  \item \textbf{The hardest part is the beginning and ending period.} At the beginning, I was new to Human Pose Estimation problem totally.
  At that time, leader requested me to do a quick review of the recent related works and make a presentation.
  I read papers every day and studied the source codes, but improve myself rapidly.
  In the last few weeks, all the ideas have been tried and our results on COCO benchmark have been stalled.
  I remember clearly that I was particularly anxious every day during that time.
  \item \textbf{Preparing the competition is totally different from doing research work.}
  In my opinion, when doing research, we should believe that one idea is useful and stick to it.
  But when preparing the competition, we may try many ideas and each idea has a end time point in order to avoid risk.
  For example, throughout the whole May, I have tried using GAN to generate new train samples. However it didn't work finally.
  So we started another attempt even though we all thought generating train data is a meaningful work. 
\end{enumerate}


%%% Local Variables:
%%% mode: latex
%%% TeX-master: "rapportM2R"
%%% End:
